
\documentclass{beamer}
\usepackage[T1]{fontenc}
\usepackage[utf8]{inputenc}

\mode<presentation> {

% The Beamer class comes with a number of default slide themes
% which change the colors and layouts of slides. Below this is a list
% of all the themes, uncomment each in turn to see what they look like.

%\usetheme{default}
%\usetheme{AnnArbor}
%\usetheme{Antibes}
%\usetheme{Bergen}
%\usetheme{Berkeley}
%\usetheme{Berlin}
%\usetheme{Boadilla}
%\usetheme{CambridgeUS}
%\usetheme{Copenhagen}
%\usetheme{Darmstadt}
%\usetheme{Dresden}
%\usetheme{Frankfurt}
%\usetheme{Goettingen}
%\usetheme{Hannover}
%\usetheme{Ilmenau}
%\usetheme{JuanLesPins}
%\usetheme{Luebeck}
\usetheme{Madrid}
%\usetheme{Malmoe}
%\usetheme{Marburg}
%\usetheme{Montpellier}
%\usetheme{PaloAlto}
%\usetheme{Pittsburgh}
%\usetheme{Rochester}
%\usetheme{Singapore}
%\usetheme{Szeged}
%\usetheme{Warsaw}

% As well as themes, the Beamer class has a number of color themes
% for any slide theme. Uncomment each of these in turn to see how it
% changes the colors of your current slide theme.

%\usecolortheme{albatross}
%\usecolortheme{beaver}
%\usecolortheme{beetle}
%\usecolortheme{crane}
%\usecolortheme{dolphin}
%\usecolortheme{dove}
%\usecolortheme{fly}
%\usecolortheme{lily}
%\usecolortheme{orchid}
%\usecolortheme{rose}
%\usecolortheme{seagull}
%\usecolortheme{seahorse}
%\usecolortheme{whale}
%\usecolortheme{wolverine}

%\setbeamertemplate{footline} % To remove the footer line in all slides uncomment this line
%\setbeamertemplate{footline}[page number] % To replace the footer line in all slides with a simple slide count uncomment this line

%\setbeamertemplate{navigation symbols}{} % To remove the navigation symbols from the bottom of all slides uncomment this line
}

\usepackage{graphicx} % Allows including images
\usepackage{booktabs} % Allows the use of \toprule, \midrule and \bottomrule in tables

%----------------------------------------------------------------------------------------
%	TITLE PAGE
%----------------------------------------------------------------------------------------

\title[Taller CR]{Taller de ChiRunning – 2do módulo} % The short title appears at the bottom of every slide, the full title is only on the title page

\author{Benjamín Juárez} % Your name
\institute[Instructor certificado] % Your institution as it will appear on the bottom of every slide, may be shorthand to save space
{%
Instructor certificado de ChiRunning \\% University of California \\ % Your institution for the title page
\medskip
\textit{cr.southamerica@gmail.com} % Your email address
}
\date{2017%\today
} % Date, can be changed to a custom date

\begin{document}

\begin{frame}
\titlepage % Print the title page as the first slide
\end{frame}

\begin{frame}
\frametitle{Esquema de entrenamiento} % Table of contents slide, comment this block out to remove it
\tableofcontents % Throughout your presentation, if you choose to use \section{} and \subsection{} commands, these will automatically be printed on this slide as an overview of your presentation
\end{frame}

%----------------------------------------------------------------------------------------
%	PRESENTATION SLIDES
%----------------------------------------------------------------------------------------

%------------------------------------------------
\section{Correr} 

\begin{frame}
\frametitle{Correr}
Con la postura aprendida y los pies y tobillos relajados está 
 todo listo para dejarse caer hacia adelante. Puede sentirse la
 liviandad de moverse sin empujar con los pies hacia adelante.
 Se busca que sea la gravedad, y no el esfuerzo muscular, lo
 que permita el movimiento del cuerpo. Con una pisada
 rápida pero de pasos cortos se puede percibir el movimiento
 de correr aún a poca velocidad.
\end{frame}

%------------------------------------------------
\section{Tranco}
%------------------------------------------------

\begin{frame}
\frametitle{Tranco}
El tranco y la velocidad de la corrida se pueden aumentar
 gracias a la inclinación, por pequeña que sea. Los ejercicios
 de esta parte apuntan a mostrar cómo una mayor movilidad
 de rotación pélvica es la que permite que los pasos se
 alarguen sin requerir mayor fuerza de piernas.
\end{frame}


\section{Cadencia}
\begin{frame}
\frametitle{Cadencia}
El impulso de la inclinación permite que las piernas hagan      
 ciclos rápidos con inercia. Una rotación de piernas más lenta
 traba ese momentuum. El ritmo de entre 170-180 pisadas
 por minuto no tiene necesidad de variar, incluso a
 velocidades mayores. Los ejercicios de comparación de
 velocidad sirven para experimentar esa diferencia.
\end{frame}

\section{Rodillas}
\begin{frame}
\frametitle{Rodillas}
Para relajar la parte más baja de las piernas se levantan los   
 pies hacia atrás sin elevar las rodillas. El ejercicio sobre
 superficie móvil también permite disminuir la fricción contra
 el suelo hacia adelante y hacia atrás. La alternancia insiste
 en la postura dinámica relajada y permite, al correr, un
 recorrido de pie más bien cı́clico antes que pendular.
\end{frame}

\section{Resumen}
\begin{frame}
\frametitle{Resumen}
Los ejercicios de esta parte buscan afinar el control sobre la
 postura, la percepción de los movimientos, aumentar la
 movilidad pélvica. La perseverancia en el entrenamiento
 permite que un nivel dé paso al siguiente. Una secuencia de
 aprendizaje recomendable irı́a en dirección a un orden de:
 forma, distancia, velocidad.
\end{frame}


\begin{frame}
\Huge{\centerline{¿Dudas?}}
\end{frame}

%----------------------------------------------------------------------------------------

\end{document} 
